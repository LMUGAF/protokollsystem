\documentclass{fssitzung}

%% Fakte the release type
%\boolfalse{public}

\anwesend(18:00){Dorothe}
\anwesend(18:00){Bernd}(19:00)
\anwesend{Conrad}(19:00)
\anwesend{Anton}
\anwesend[leitung](18:00){Ludwig}
\anwesend[protokoll]{Petra}(19:00)
\anwesend[protokoll]{Paul}

\beginn{18:07}
\ende{20:00}

%% Post
\post{Ein kurzer Post Punkt}
\post{Ein weiterer kurzer Post Punkt}
\post[Kurzer Titel]{Ein kurzer Post Punkt mit Titel}

\begin{POST}{Langer Titel}
	Hier gibt es so viel zu besprechen,
	
	das man sogar mehrere Absätze benötigt.
	
	\todo{Auf anfrage antworten}(Bernd Nörgel)
\end{POST}

%% Diskussion
\begin{DISKUSSION}{Pro und Contra}
	Winführung in die Diskusion
	\pro{Punkte}
	\pro{die}
	\pro{dafür}
	\pro{sprechen}
	
	\con{und}
	\con{welche}
	\con{dagegen}
	
	\pro{noch einer dafür}
	
	Etwas Text.
	\pro{Kann man schon machen}

	\todo{Wichtige dinge tun}(Bernd Nörgel)[gestern]
	\todo{Wichtige dinge nicht tun}(Ludwig)
	\todo[done]{Make me a sandwich}(Bernd Nörgel)
\end{DISKUSSION}

%% Anträge
\antrag{angenommen}{Reiche Eltern für alle}{Hier gibt es nicht viel zu sagen}
\transponderAntrag{Bernd Nörgel}
\accountAntrag{Bernd Nörgel}

\begin{ANTRAG}{abgelehnt}{Freibier für alle}
	Eine hitzige diskusion entwickelt sich,
	
	weswegen man mehrere Absätze benötigt:
	
	\con{Meh}
\end{ANTRAG}


%% Berichte
\bericht{Kurzer Bericht ohne Infos}
\bericht[Kurzer Bericht]{mit Infos}
\berichtFakRat{16}

\begin{BERICHT}{Ausführlicher Bericht}
	Viel
	
	Text.
\end{BERICHT}

%% Termine
\termin{2014-1-1}{Ich kann Termine sortieren 1}
\termin{2014-1-3}{Ich kann Termine sortieren 3}
\termin{2014-1-2}{Ich kann Termine sortieren 2}
\termin{2014-1-4}{Ich kann Termine sortieren 4}

%% WAS
%% Star-Versionen sollen komplett intern sein (nicht angezeigt wenn public
%% version)
\was*{Kurzer WAS Punkt mit stern}
\was{Kurzer WAS Punkt ohne stern}
\was*[Kurzer Titel]{Kurzer WAS Punkt}

\begin{WAS}*{Langer Titel}
	Langer
	
	Punkt.
\end{WAS}


%% Has to be the last command
\sitzung{12.4.2014}
